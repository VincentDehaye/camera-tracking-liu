\documentclass[twocolumn]{article}
\usepackage[utf8]{inputenc}

\title{Project report}
\author{{Martin Estgren \texttt{<mares480>}} \\
        {{Vincent Déhaye \texttt{<vinde799>}} \\
        {Sebastian Maghsoudi \texttt{<sebma654>}} \\~\\
        {Linköping University (LiU), Sweden}}}
\date{September 2017}

\begin{document}

\maketitle

\tableofcontents

\newpage

\section{Introduction}
 This section strives to introduce the reader to the subject. It also satisfies the need for clear goals in order for the reader to get a clear understanding of the knowledge that one will have after reading this paper. Here will also exist an explanation of the contents of this paper.
\subsection{Background}
\subsection{Goals and research questions}
\subsection{Content}

\section{Theory}
 This section has two purposes, the first motive is to lay a good theoretical groundwork for the reasoning that will result in a relevant conclusion. The second purpose is to highlight definitions that is necessary for any potential readers to understand before completely being able to grasp the subject. 

\subsection{Detection}
 Detection will here be defined as the process of detecting relevant objects and highlight them somehow for future processes, this is commonly done by assigning bounding boxes to the detected object. The detection is done frame by frame with different methods. Since this paper reflects and revolves around the attempt to apply some algorithms to obstacle avoidance, it is perforce to define relevant obstacles. In this case the focus is to avoid humans which gives the research an angle where there may be more simple to find already proven methods for the specific scenario. Using different visual recognition application which was slightly modified to detect pedestrians, Dollár et al.\cite{dollar2014fast}
 was able to display their efficiency in the relevant context. An option to the, by Dollárs\cite{dollar2014fast} proposed, methods is convolutional neural network (CNN) based detectors.       
\subsubsection{FrCNN}
 
 
 
\subsection{Tracking}
 Tracking is here defined as the process of tracking detected objects frame to frame and assigning an identification to them. There are multiple options for this yet this paper is in a circumstantial position to bring up only some of these

\subsubsection{SORT}
 Simple Online and Real Tracking, in this document henceforth refereed to as SORT, is presented by Bewly et al.~\cite{Bewley2016_sort}. The method was implemented to solve the problem of multiple object tracking\footnote{https://motchallenge.net/} (MOT). SORT views the problem as a so called association problem where the tracking is equivalent to associate the same detected objects in different frames. The authors claim that in order to achieve high accuracy it is necessary several known methods since most existing accurate methods are to slow for real time application.   

\section{Method}

\section{Implementation}

\section{Result}

\section{Conclusion}

\bibliography{source}{}
\bibliographystyle{plain}

\end{document}
